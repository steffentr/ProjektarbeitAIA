\documentclass[a4paper,10pt]{article}
\usepackage[utf8]{inputenc}
\usepackage{tocloft}
\usepackage{nomencl}
\usepackage{pdfpages}
\usepackage{bm}
%opening
\title{Investigations on one-way coupling effects of particle-laden decaying isotropic turbulent flows}
\author{Julian Stemmermann, Steffen Trienekens, Christian Soika}
\date{Aachen, 3rd of July 2017}

\begin{document}

\maketitle

\pagebreak

\tableofcontents{}
 
\pagebreak

\section{Nomenclature}
\makenomenclature
\printnomenclature
\pagebreak

\section{Introduction}
\pagebreak
\section{Mathematical models}
\subsection{Single-phase flow} %Christian
In this section the mathematical basics for understanding and simulating turbulent flows are discussed. However, it should be pointed out that this is no
complete treatise of the mathematical and physical basics. The reader can achieve further insight on this topic by looking at different books and papers, 
e.g. Pope (2000).
\newline
The Navier-Stokes-Equations are of great importance for understanding turbulent phenomena. This set of equations exists in forms for compressible and
incompressible fluids. The starting point is the realization that the force on a fluid-volume equals the divergence of the cauchian stress tensor
integrated over the whole volume.

\begin{equation}
  \vec F = \int_{\mathrm{A}}\vec s\mathrm{d}A = \int_{\mathrm{V}}\ \mathrm{div}(\bm{\sigma})\mathrm{d}V
\end{equation}


Christoph Siewert:
-2.1 bis 2.6
Stephan Fritz:
-Navier-Stokes-Gleichungen (Anhang B)
\pagebreak
\subsection{Particle dynamics} %Steffen
Siewert:
-3.1a-3.14 (spherical particles) OHNE GRAVITATION
Stokes Drag/Stokes Coefficient
Filterung (Fritz) ->Viskositaet durch numerischen Fehler, Smagorinksy nicht benutzen
\pagebreak
\section{Numerical methods} %Julian
Pojektion (noComputationalParticles)
implizite LES (Motivation fuer LES -> Pope Chapter 9, Bild 9.4), DNS
\pagebreak
\section{Results}
Graphen (particleFree rot, Laden gruen)
\pagebreak
\section{Conclusion}
\pagebreak
\section{Bibliography}
\pagebreak

\end{document}
